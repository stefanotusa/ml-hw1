
\documentclass[letterpaper, 11pt]{article}
\usepackage[margin=1in]{geometry}

% Set the typeface to Times Roman
\usepackage{times}

%\usepackage{hyperref}
\usepackage{amsfonts}%
\usepackage{amssymb}%
\usepackage{amsthm}% allows theoremstyles (see below) and provides a proof environment
\usepackage{bm}
\usepackage{relsize}
\usepackage{graphicx}
\usepackage{caption}
\usepackage{epstopdf}
\usepackage{amsmath}
\usepackage{tikz}
\usepackage{hyperref}
\usetikzlibrary{trees,arrows}
\usepackage{cite}
\usetikzlibrary{decorations}
\usetikzlibrary[decorations]
\usepgflibrary{decorations.pathmorphing}
\usepgflibrary[decorations.pathmorphing]
\usetikzlibrary{decorations.pathmorphing}
\usetikzlibrary[decorations.pathmorphing]
\usepackage{booktabs}
\usepackage[authoryear]{natbib}
\usepackage{subcaption}
\usepackage{pseudocode}
%\usepackage{float}
\usepackage{verbatim} %% for commenting blocks
\usepackage{cancel}

\bibpunct{(}{)}{,}{}{}{;} %% added this to make \citep{x} use parentheses

\newcommand{\problemAnswer}[1]{%#1% Defines the problem answer command with the content as the only argument
\noindent\framebox[0.95\columnwidth][c]{\begin{minipage}{0.92\columnwidth}\color{blue}{#1}\end{minipage}} % Makes the box around the problem answer and puts the content inside
}

%% independence symbol and expectation operator %%
\newcommand\independent{\protect\mathpalette{\protect\independenT}{\perp}}
\def\independenT#1#2{\mathrel{\rlap{$#1#2$}\mkern2mu{#1#2}}}

\DeclareMathOperator{\circlearrow}{\hbox{$\circ$}\kern-1.5pt\hbox{$\rightarrow$}}
\DeclareMathOperator{\circlecircle}{\hbox{$\circ$}\kern-1.2pt\hbox{$--$}\kern-1.5pt\hbox{$\circ$}}

\DeclareMathOperator{\an}{an}
\DeclareMathOperator{\pa}{pa}
\DeclareMathOperator{\ch}{ch}
\DeclareMathOperator{\pre}{pre}
\DeclareMathOperator{\de}{de}
\DeclareMathOperator{\nd}{nd}
\DeclareMathOperator{\sib}{sib}
\DeclareMathOperator{\dis}{dis}
\DeclareMathOperator{\mb}{mb}
\DeclareMathOperator{\doo}{do}
\DeclareMathOperator{\odds}{\text{OR}}
\DeclareMathOperator*{\argmax}{arg\,max}
\DeclareMathOperator*{\argmin}{arg\,min}
\definecolor{lemon}{RGB}{ 242, 200,  24}
\def\ci{\perp\!\!\!\perp}
\newcommand{\E}{\mathbb{E}}
\newcommand{\G}{\mathcal{G}}

\newcommand\indep{\protect\mathpalette{\protect\independenT}{\perp}}
\def\independenT#1#2{\mathrel{\rlap{$#1#2$}\mkern2mu{#1#2}}}

\newtheorem{Lma}{Lemma}
\newtheorem{Thm}{Theorem}
%%%%%%%%%%%%%%%%%%%%%%%%%%%%

\title{CS 476/676 (Spring 2021): Homework 1}

\author{}

\date{Due: Feb 3, 2021 at 11:59pm EST}

\begin{document}

\maketitle

\setlength{\parindent}{0em}
\setlength{\parskip}{0.8em}

\Large\textbf{Name:} \underline{\hspace{30pt} \color{blue} Stefano Tusa \hspace{30pt}}
\vspace{1em}

\textbf{Instructions}: This homework requires some open-ended questions, short proofs, and
programming. This is an individual assignment, not group work. Though you may
discuss the problems with your classmates, you must solve the problems and
write the solutions independently. As stated in the syllabus, copying code
from a classmate or the internet (even with minor changes) constitutes
plagiarism. You are required to submit your answers in pdf form (use \LaTeX)
in a file called \texttt{<your-JHED>-hw1.pdf} to Gradescope. (Note: for those who not familiar with \LaTeX, you can visit \url{https://www.overleaf.com} to quickly start by using the provided .tex template file.) \textcolor{red}{The code for the programming assignment will be handed in separately on
github; see the instructions under Problem 8.}
Late submissions will be penalized, except in extenuating circumstances such
as medical or family emergency. Submissions submitted 0-24 hours late will be
penalized 10\%, 24-48 hours late by 20\%, 48-72 hours late by 30\%, and later
than 72 hours by 100\%. Late days may be used (if available) to avoid these penalties. Problems 1 and 2 are worth 5 points each, Problem 3 is worth 10 points, and Problem 4 is worth 15 points, for a total of 35 points.

\vspace{1em}

{\Large\textbf{Problem 1}}

This is an open-ended question to get you a head start on thinking about what you would like to do for your final project. Write a brief description of the kinds of data you have worked with (if you have not worked with data before, it is ok to say so) and would like to work with in this course. What questions did you/would you like to answer with such data? There are no right or wrong answers; a paragraph or two is sufficient.

\problemAnswer{
I've worked with assorted datasets in CS475: Machine Learning,
 but no particular one stands out in my memory. I did compile a
small dataset of consisting of song lyrics for Billboard's 
then-current Top 100 Songs  for a personal project that was then
put through some standard NLP pre-processing steps.

In this class, I would like to work with some kind of social
 network data, or perhaps more advanced listening analytics from
a streaming service such as Spotify or Apple Music. 
Generally, I am interested more in sociological,
cultural, or psychological data.
\vspace*{150pt}
}

\vspace{1em}
{\Large\textbf{Problem 2}}

Prove that $X \ci Y \mid Z$ implies that $\odds(X, Y \mid Z)=1$ for all $x, y, z$.

%
\problemAnswer{
\[\odds(X, Y \mid Z)=\frac{p(X=x \mid Y=y, Z)}{p(X=x_0 \mid Y=y, Z)} \times \frac{p(X=x _0 \mid Y=y_0, Z)}{p(X=x \mid Y=y_0, Z)} \] \\ 

\[= \frac{p(X=x \mid Z)}{p(X=x_0 \mid Z)} \times \frac{p(X=x _0 \mid Z)}{p(X=x \mid Z)} = 1 \] \\

\vspace*{150pt}
}
%

\vspace{1em}
{\Large\textbf{Problem 3}}

Consider three sets of random variables $X$, $Y$, and $Z.$ Assume that $X$ and $Y$ are binary random variables and the relations between $X$, $Y$, and $Z$ can be modeled via a linear logistic regression (i.e., no interaction terms.) That is,
%
\begin{align*}
p(X=1 &\mid Y, Z) = \frac{1}{1 + \text{exp}\big(-(\beta_0 + \beta_1 Y + \beta_2 Z)\ \big)}.
\end{align*}
%
Let the reference values of $X$ and $Y$ be $x_0=0$ and $y_0=0$ respectively. Prove 
%
\begin{align*}
\odds(X, Y \mid Z) = \begin{cases}
	\text{exp}(\beta_1), & \text{if $X=Y=1$}\\
	1, & \text{otherwise.}
\end{cases}
\end{align*}
%

\problemAnswer{
\[\odds(X, Y \mid Z)=\frac{p(X=x \mid Y=y, Z)}{p(X=x_0 \mid Y=y, Z)} \times \frac{p(X=x _0 \mid Y=y_0, Z)}{p(X=x \mid Y=y_0, Z)} \] \\

\[= \frac{\frac{1}{\cancel{1 + \text{exp}\big(-(\beta_0 + \beta_1 X + \beta_2 Z)\ \big)}}}{\frac{\text{exp}\big(-(\beta_0 + \beta_1 X + \beta_2 Z)\ \big)}{\cancel{1 + \text{exp}\big(-(\beta_0 + \beta_1 X + \beta_2 Z)\ \big)}}} \times \frac{\frac{\text{exp}\big(-(\beta_0 + \beta_1 X + \beta_2 Z)\ \big)}{\cancel{1 + \text{exp}\big(-(\beta_0 + \beta_1 X + \beta_2 Z)\ \big)}}}{\frac{1}{\cancel{1 + \text{exp}\big(-(\beta_0 + \beta_1 X + \beta_2 Z)\ \big)}}}\] \\

\[ = \frac{1}{\exp{\beta _0}}

\]
\vspace*{150pt}
}

\vspace{1em}
{\Large\textbf{Problem 4}}

Implement a method/function which takes data in the form of a comma separated file (csv) and returns a point estimate for the odds ratio via logistic regression, and confidence intervals via bootstrap for the same. We recommend using the \texttt{statsmodels} package in Python for fitting any regression models in this course -- the reason for this is because \texttt{sklearn} does not allow you to (easily) turn off regularization. Regularization is useful when considering out-of-sample prediction, but will bias your estimate of the statistical parameters we are interested in computing.

Use your function to compute point estimates and confidence intervals for \linebreak $\odds(\text{Mortality}, \text{Opera})$ and $\odds(\text{Mortality}, \text{Opera} \mid \text{Income})$ from the data provided in \texttt{data.txt}. Set the number of bootstraps to $200$ when computing your confidence intervals. Based on the numbers you have computed, would you say\

a) $\text{Mortality} \ci \text{Opera}$?\\
b) $\text{Mortality} \ci \text{Opera} \mid \text{Income}$?

Provide a brief comment on how seriously the above numbers you computed and the implied dependence/independence relations should be interpreted causally.

Report the numbers you computed and answers to the above questions in the PDF that you turn in. Hand in your Python code separately. Please make sure your variables and functions are aptly named, and your code is clear and readable, with comments where appropriate. We may try to execute your code on a new dataset as a test case, but don't worry too much about making your code robust in terms of error checking and adversarial inputs -- this new dataset will look very similar to the one we have provided.

A brief explanation of the data and the variables. The data is simulated, but heavily inspired by a longitudinal study on the health benefits of engaging in the arts -- going to the opera, theatre, museum etc.
\begin{itemize}
	\item \texttt{age}: individual's age at beginning of the study
	\item \texttt{income}: individual's income at beginning of the study
	\item \texttt{college\_degree}: did the individual hold a college degree at the beginning of the study ($1=$ yes, $0=$ no)
	\item \texttt{opera}: was the individual attending the opera regularly (once a month or more) at the beginning of the study ($1=$ yes, $0=$ no)
	\item \texttt{mortality}: did the individual die during the 5-year period of the study ($1=$ yes, $0=$ no)
\end{itemize}

\problemAnswer{
Write your answer here. 
\vspace*{300pt}
}



\end{document}

